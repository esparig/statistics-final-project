Se analizaron los tiempos de ejecución de 4 ejecuciones independientes y se encontró una variabilidad significativa en los tiempos de ejecución para diferentes combinaciones de parámetros. Al aplicar el logaritmo natural a los valores de la media y generar un histograma, se observó que la distribución no es normal y que hay múltiples subpoblaciones en los datos. La prueba de bondad de ajuste de Kolmogorov-Smirnov confirmó que los datos no siguen una distribución normal.

Se realizó un análisis de correlación entre el tiempo de ejecución de un producto matriz-matriz y diferentes parámetros que afectan su rendimiento. Los parámetros más relevantes en cuanto a correlación con el tiempo de ejecución resultaron ser MWG y NWG, con una relación positiva moderada. Los parámetros MDIMC y NDIMC también resultaron relevantes, con una relación negativa moderada. Los demás parámetros obtuvieron una relación más débil con el tiempo de ejecución.

Se estudió la distribución de los datos. El histograma de los logaritmos de las ejecuciones sugería que los datos son una mezcla de varias distribuciones normales. Se generó un modelo de mezcla gaussiana puede ser utilizado para representar los datos y puede ser usado para clustering, detección de valores atípicos y estimación de densidad.