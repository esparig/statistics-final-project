Para llevar a cabo el análisis estadístico, se utilizó el entorno de Google Colab, que proporciona una plataforma en línea para ejecutar código Python en un Jupyter Notebook. Se utilizaron varias librerías especializadas en análisis estadístico, como NumPy \citep{harris2020array}, Pandas\citep{mckinney2010data}, SciPy\citep{2020SciPy-NMeth}, Scikit-learn\citep{scikit-learn} y Matplotlib\citep{Hunter2007} entre otras. 

El procedimiento de análisis estadístico se dividió en varias etapas. En primer lugar, se realiza una exploración general de los datos para comprender mejor su estructura y distribución. A continuación, se llevó a cabo un análisis de correlación para determinar si existía alguna relación entre los parámetros que se analizarían. Finalmente, se generó un modelo para la representación de los datos.