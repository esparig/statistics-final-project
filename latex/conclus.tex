En resumen, el análisis estadístico de los tiempos de ejecución y los parámetros que afectan su rendimiento ha permitido obtener información valiosa sobre la estructura de los datos y las relaciones entre las variables. Se ha encontrado una variabilidad significativa en los tiempos de ejecución para diferentes combinaciones de parámetros y se han identificado los parámetros más relevantes en cuanto a correlación con el tiempo de ejecución. Además, se ha utilizado un modelo de mezcla gaussiana para representar la distribución de los datos y se ha demostrado su utilidad para clustering, detección de valores atípicos y estimación de densidad. Este análisis estadístico demuestra el interés y la importancia de utilizar técnicas estadísticas avanzadas para obtener información útil a partir de los datos.