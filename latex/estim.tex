Se han llevado a cabo distintas experimentos para obtener un modelo que estime adecuadamente el tiempo de ejecución en función de los parámetros de entrada. 

Aunque era de esperar que una simple regresión linean no funcionase adecuadamente con nuestros datos, se realizó el experimento como ejercicio académico. Para ello se ha ajustado un modelo de regresión lineal y se ha realizado un paso de validación mediante la técnica de validación cruzada. 

La validación cruzada es una técnica utilizada para evaluar la calidad del ajuste de un modelo a un conjunto de datos especialmente útil cuando se dispone de un conjunto de datos limitado. Consiste en dividir el conjunto de datos en varias particiones y utilizar cada una de ellas para evaluar el modelo ajustado con las demás particiones. Esto permite obtener una estimación más precisa de la capacidad del modelo para generalizar a nuevos datos. En nuestro experimento hemos utilizado 5 particiones.

